% Resume autogenerated using resume-latex.xsl

\documentclass[a4]{resume}
\usepackage{CJK}
\usepackage{paralist}
\definecolor{ruleendcolor}{rgb}{0.6, 0.6, 0.6}

\email{baohaojun@gmail.com}
\phone{18610314439}
\webpage{http://baohaojun.github.io}

\begin{document}
\begin{CJK*}{UTF8}{simsun}
\author{包昊军}
\CJKtilde
\maketitle

\section{教育背景}

\affiliation[本科,控制理论与工程]
            {浙江大学}
            {1997年9月-2001年7月}

\affiliation[硕士,控制理论与工程]
            {中科院自动化所}
            {2001年9月-2004年7月}

\section{工作经验}

\affiliation[资深软件工程师]
            {Marvell北京}
            {2011年11月 \~ 今}

\begin{compactitem}

  \item 在Marvell手机整体方案团队工作。\par

  \item 前期负责工厂生产工具的开发,去工厂产线支持工具应用,并进行评估、
    改进。

  \item 负责一部分bsp开发、维护:OBM、Uboot、Audio。

\end{compactitem}

\affiliation[资深软件工程师]
            {RayzerLink/Letou}
            {2010年3月 \~ 2011年10月}

\begin{compactitem}
  \item 这是两家致力于设计、生产基于Android系统平板电脑的公司,均由我在
    播思通讯时的同事创办。\par

  \item 负责基于Nvidia Tegra2芯片的平板电脑底层软件开发,主要包括Linux
    Kernel bringup,驱动(Touch、Lcd、Sensors),Hal(硬件抽象层)的开
    发等工作。

  \item 指导bsp新同事底层开发工作。
\end{compactitem}

\affiliation[高级软件工程师]
            {播思通讯 (\url{http://www.borqs.com})}
            {2008年11月 \~ 2010年3月}

\begin{compactitem}

        \item 播思是开发基于Google Android平台的OMS系统的手机软件公司,我在其Tools组工作。\par

        \item 设计并实现 Borqs Engineering Tool (C++,VC6,MFC),一个手机开发、测试工具。

        \item 设计并实现fb2bmp ( C,arm-linux-gcc)。从手机的frame
          buffer中截屏的工具。

        \item 实现Engineering Tool Library,方案客户可以基于该工具库进
          行二次开发。

        \item 实现Service Tool。基于Engineering Tool Library开发,提供给客户的维修中心使用.

        \item 设计并实现SerialPort Usblan Passthrough Tool (Python)。一
          个串口、以太网转换工具。

        \item 开发烧机工具 (VC++)和打包工具 (Bash script,Python)。

        \item 指导新工程师的软件开发,开源软件工具的使用。指导实习生用
          Python和PyQt开发跨平台的Borqs Engineering Tool.

\end{compactitem}

\affiliation[软件工程师]
            {摩托罗拉,移动设备部/全球软件中心}
            {2005年9月-2008年9月}

\begin{compactitem}

        \item 手机多媒体软件自动调试工具开发

        \item 手机多媒体软件开发

\end{compactitem}

\affiliation[软件工程师]
            {麒麟软件}
            {2004年10月-2005年9月}
\begin{compactitem}
  \item 企业集成应用软件测试
\end{compactitem}

\affiliation[自由软件项目]
            {Win32,Cygwin,GNU/Linux \par (\url{http://windows-config.googlecode.com})}
            {2005年5月 \~ 今}

\begin{compactitem}

  \item skeleton-complete.el,一个Emacs下的补齐工具(Emacs-lisp)。

  \item CrossDict,一个Android下的英文字典软件,在Google Play上发布
    (Java,Android) 。

  \item org-jira.el,一个Emacs下用org-mode来进行Jira开发流程管理的工具
    (Emacs-lisp)。

  \item beagrep,一个结合搜索引擎的源代码grep工具,一秒钟grep两G代码
    (C\#,Perl)。

  \item sdim,一个跨所有主流平台(Win32/Linux/Mac OS基至Emacs)的输入法
    (Python,C++,ObjC,Emacs-lisp)。

  \item scim-fcitx,一个GNU/Linux下的输入法,基于scim和fcitx移植(C++,GNU/Linux)。

  \item 其他一些较小的脚本/程序,均放在
    \url{https://github.com/baohaojun}下用git管理。

\end{compactitem}


\section{计算机技能}


\begin{compactitem}

      \item 编程语言: C++,C,Python,Bash,一点Perl,一点Emacs Lisp,
        一点ObjC,一点Java。\par

      \item Emacs。\par

      \item GNU/Linux,Cygwin。\par

\end{compactitem}

\end{CJK*}
\end{document}
